\PaperTitle{论玄学中~$\bm{ SO(10)} $~自发对称性破缺} % Article title


\Authors{李雷\textsuperscript{1}*, 韩梅梅\textsuperscript{2}} % Authors
\affiliation{
	\quad
	\textsuperscript{1}\textit{北京大学物理学院}
	\qquad
	\textsuperscript{2}\textit{北京大学物理学院理论研究所}
	\qquad
	*\textbf{通讯作者}: lilei@pku.edu.cn
} % Author affiliation

\Abstract{
	\phantom{田田}玄学又称新道家,是对《老子》、《庄子》和《周易》的研究和解说,产生于魏晋。玄学是中国魏晋时期到宋朝中叶之间出现的一种崇尚老庄的思潮。也可以说是道家之学的一种新的表现方式,故又有新道家之称。其思潮持续时间自汉末起至宋朝中叶结束。玄学是魏晋时期取代两汉经学思潮的思想主流。 玄学即“玄远之学”,它以“祖述老庄”立论,把《老子》、《庄子》、《周易》称作“三玄”。道家玄学也是除了儒学外唯一被定为官学的学问。
}


\Keywords{\phantom{田田}玄学\quad玄学\quad玄学} % 如不需要关键词可直接删去花括号中内容

